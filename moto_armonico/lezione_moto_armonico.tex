\documentclass{article}
\usepackage[T1] {fontenc}
\usepackage[italian]{babel}
\usepackage{cite}

\usepackage{graphicx}
\graphicspath{ {./images/} }
\usepackage{float}

% Per i link nell'indice
\usepackage{hyperref}

\title{Lezioni sul Moto Armonico}
\author{Daniele Meloccchi, Agnese Montanaro, Matteo Savatteri}

\begin{document}
\maketitle

\tableofcontents

\section{Introduzione}
Questo documento presenta un ciclo di lezioni sul moto
armonico, indirizzate a una classe 4\textsuperscript{a}
di liceo scientifico italiano e preparate secondo il
\emph{Modello di Istruzione 5E} \cite{bybee2009bscs}.
Questo metodo di natura inquiry prevede cinque fasi:
\emph{Engage}, \emph{Explore},
\emph{Explain}, \emph{Extend}, \emph{Evaluate}.

\section{Engage}
Per far scoprire ai ragazzi l'esistenza del moto armonico e
la sua descrizione, occorre innanzitutto attirare la loro
attenzione portando in classe un sistema che mostri un moto
oscillatorio. A questo scopo mostriamo agli studenti un modellino di
\emph{pendulum wave} (Figura \ref{fig:pendulum_wave}): una
serie di pendoli agganciati a un sostegno in legno mobile.
Ciascun pendolo è costituito da un filo di stessa lunghezza
e da un pesetto di uguale massa. 
Facciamo oscillare contemporaneamente tutti i pendoli:
tutti hanno lo stesso periodo. 
Modifico la lunghezza dei pendoli sollevando un'estremità del
sostegno e li faccio oscillare contemporaneamente. I periodi
dei diversi pendoli si modificano in modo che si formi un'onda
longitudinale (guardando il sistema dall'alto).

\begin{figure}
\centering
  \includegraphics[width=0.32\textwidth]{pendulum_wave3}
  \includegraphics[width=0.32\textwidth]{pendulum_wave2}
  \includegraphics[width=0.32\textwidth]{pendulum_wave1}
  \caption{Modello di pendulum wave.}
  \label{fig:pendulum_wave}
\end{figure}

\section{Explore}
Dopo aver catturato l'attenzione degli studenti col nostro engage, conduciamo i ragazzi all'interno della
materia offrendo loro altri moti.
In particolare sottoponiamo la loro attenzione ai seguenti tre esperimenti:

\begin{itemize}
\item un pendolo che
oscilla;
\item un cilindretto che si muove lungo un tracciato parabolico;
\item una pallina che si muove lungo una traiettoria a V.
\end{itemize}

Il pendolo del primo esperimento consiste di un nastro alla cui estremità è legato un peso (un
sassolino il più possibile puntiforme) e che viene fatto oscillare manualmente.

Il secondo, invece, consiste di un cilindro metallico, che messo inizialmente alla estremità di un cuscino
lombare, avente una sella a forma parabolica, viene lasciato cadere lungo questa conca. Il peso oscilla
lungo questa regione fino a che, dopo poche oscillazioni, si ferma.

L'ultimo esperimento, infine, consiste di un moto simile al precedentem con la differenza che la traiettoria
viene realizzata accostando due piani inclinati (costituiti da mattoni e piastrelle), facendo sì che i due
abbiano come punto di contatto le loro estremità finali, così da formare una sorta di “V”. Il punto
materiale è una pallina di spugna.

Lasciamo indagare gli studenti circa i fenomeni fisici proposti e chiediamo di fare una classificazione
dei moti osservati in base alle eventuali somiglianze.
Chiediamo anche di tracciare, scegliendo un opportuno sistema di riferimento, un grafico della forza in
funzione della posizione, $F_x(x)$\cite{barbieri2015good}.

\section{Explain}
Nella lezione seguente possiamo approcciare la fase di explain.
Essa dovrebbe essere svolta sotto la guida dell'insegnante e 
sulla base dell'esperienza che gli studenti hanno acquisito nelle
fasi precedenti. \`E fondamentale che a questo punto gli studenti
condividano le spiegazioni dei fenomeni che essi stessi
hanno sviluppato \cite{duran20045e}.
Confrontiamo i grafici $x(t)$ e $a(x)$ ottenuti utilizzando il software di
analisi video \emph{tracker}\footnote{\url{https://physlets.org/tracker/}}.

\begin{figure}[H]
\centering
  \includegraphics[width=\textwidth]{pendolo_x_t}
  \includegraphics[width=\textwidth]{pendolo_a_x}
  \caption{Grafici di $x(t)$ e $a(x)$ per il pendolo semplice.}
  \label{fig:pendulum}
\end{figure}

\begin{figure}[H]
\centering
  \includegraphics[width=\textwidth]{cilindro_rotaia_x_t}
  \includegraphics[width=\textwidth]{cilindro_rotaia_a_x}
  \caption{Grafici di $x(t)$ e $a(x)$ per il cilindro su rotaia parabolica.}
  \label{fig:rotaia_parabolica}
\end{figure}

\begin{figure}[H]
\centering
  \includegraphics[width=\textwidth]{rotaia_galileo_x_t}
  \includegraphics[width=\textwidth]{rotaia_galileo_a_x}
  \caption{Grafici di $x(t)$ e $a(x)$ per la pallina su rotaia a "V"}
  \label{fig:rotaia_galileo}
\end{figure}

Si può evincere che ci sono delle similitudini tra i grafici dell'accelerazione in
funzione della posizione del pendolo (Figura \ref{fig:pendulum}) e della rotaia parabolica
(Figura \ref{fig:rotaia_parabolica}), mentre il grafico della posizione in funzione
del tempo delle due rotaie (Figure \ref{fig:rotaia_parabolica} e
\ref{fig:rotaia_galileo}), di primo acchito, sembrano identici. Tuttavia non è così.
Infatti il cilindro che oscilla lungo una curva compie lo stesso moto di un pendolo e la traiettoria
$x(t)$ è rappresentata da delle sinusoidi, mentre il grave che cade lungo un piano inclinato obbedisce alla
legge dei moti in caduta libera, con accelerazione costante pari a $g\sin(\alpha)$, con $\alpha$ angolo d'inclinazione del
piano e pertanto la traiettoria $x(t)$ di quest'ultimo è costituito da rami di parabole di concavità opposta
collegati tra loro.
Pertanto, sicché la pallina compie dapprima un moto discendente e successivamente ascendente lungo i due
piani inclinati, il grafico di $a(x)$ mostra una discontinuità di tipo salto nell'origine tra due valori costanti di segno
opposto.

Per quanto visto, dunque, un moto armonico è caratterizzato dal fatto che in un intorno del suo punto di
equilibrio (in tutti i nostri esperimenti tale punto è l'origine) il grafico $F_x(x)$ (che è qualitativamente identico a
quello di $a(x)$ a meno di una costante moltiplicativa) è uguale ad una retta di coefficiente angolare negativo:

\begin{equation}
F_{x}(x)=-kx, \quad x \in U(0),\quad k>0
\end{equation}

Per quanto riguarda invece i grafici della posizione in funzione del tempo ottenuti con Tracker spieghiamo l'origine dell'andamento sinusoidale.

Abbiamo scoperto che il moto del pendolo è un moto armonico perchè segue la legge appena trovata. A partire da questa osservazione si può comprendere la legge oraria. 

Disegnamo un pendolo e tracciamo una circonferenza di raggio R, pari alla lunghezza del pendolo,
e di centro il vertice del pendolo (Figura \ref{fig:costruzione_moto_circolare}). La traiettoria
del pesetto è lungo un arco di circonferenza, per questo ha senso fare questo disegno schematico.
Avendo precedentemente affrontato il moto circolare gli studenti sanno che il modulo
dell'accelerazione centripeta è dato da: 

\begin{equation}
a = \omega^2R
\end{equation}

\begin{figure}[H]
\centering
  \includegraphics[width=0.60\textwidth]{costruzione_moto_circolare}
  \caption{Costruzione geometrica per spiegare la legge oraria del moto armonico.}
  \label{fig:costruzione_moto_circolare}
\end{figure}

Dalla costruzione  e dalle conoscenze goniometriche si trova: 

\begin{equation}
a_x(\theta) = -\omega^2 R \sin(\theta) \quad\quad x(\theta) = R \sin(\theta)
\end{equation}

Quidi accelerazione e posizione sono proporzionali tra loro. 
Mentre la velocità risulta essere: 

\begin{equation}
v_x(\theta) =  v \cos(\theta) = \omega R \cos(\theta) 
\end{equation}

Dallo studio del moto circolare sappiamo che la velocità angolare $\omega$ è
data dal rapporto tra l'angolo percorso e il tempo impiegato, pertanto $\theta$
può essere riscritto come: $\theta = \omega t$. 
Abbiamo ottenuto che la legge oraria del pendolo, e in generale del moto armonico,
è della forma:

\begin{equation}
x(t) = R \sin(\omega t) 
\end{equation}

come si osservava dai grafici di ottenuti con Tracker.

\section{Extend}
Ora lo studente è pronto a mettere in pratica le conoscenze
acquisite, estendendo la sua comprensione dell'argomento
sulla base di esse. Una competenza molto utile da acquisire
è quella di riconoscere l'armonicità di un moto, basandosi
semplicemente sull'osservazione del fenomeno.

Un corpo oscilla armonicamente nell'intorno di un punto
di equilibrio solo se \cite{giliberti2014detecting}:

\begin{itemize}
\item Il punto di equilibrio $x=0$ è \emph{stabile};
\item La funzione $F_{x}$ è continua;
\item La funzione $F_{x}$ è differenziabile;
\item $\frac{\mathrm{d}F_{x}}{\mathrm{d}x}(0) \neq 0$
\end{itemize}

Da queste considerazioni possiamo dedurre che un moto
è armonico se ha un punto di equilibrio e se appare
\emph{liscio} e \emph{isocrono}.
Esiste un trucco che permette di verificare l'isocronia
nel caso di alcuni moti: ascoltare il rumore prodotto
dal moto e verificare che la sua frequenza non cambia al
variare dell'ampiezza delle oscillazioni.

Portiamo dunque gli studenti in laboratorio, dove si divideranno
a gruppi di tre o quattro e proveranno a progettare un esperimento
che mostri o neghi l'armonicità di un certo moto.
Gli studenti dovranno avvalersi di strumenti di misura,
come cronometro e metro, per valutare l'isocronia del
moto scelto.

\section{Evaluate}
\underline{TODO}

\bibliography{bibliografia}{}
\bibliographystyle{plain}

\end{document}
