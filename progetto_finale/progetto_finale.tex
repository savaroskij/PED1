\documentclass{report} \usepackage[T1]{fontenc} \usepackage[italian]{babel}
\usepackage{color}
\usepackage[type={CC},modifier={by-sa},version={4.0},]{doclicense}
\usepackage{cite}

\usepackage{graphicx}
\graphicspath{ {./media/images/} }
\usepackage{float}

\usepackage{hyperref}

\title{Title} \author{Daniele Melocchi\\Agnese Montanaro\\Matteo
Savatteri}

\begin{document}
\maketitle
\setcounter{page}{2}

Copyright Daniele Melocchi, Agnese Montanaro, Matteo Savatteri -
\the\year \doclicenseThis \thispagestyle{empty}

\tableofcontents

\chapter{Introduzione}
Questo documento presenta un ciclo di lezioni svolte nella modalità
della didattica a distanza (\emph{DAD}) e
indirizzate ad una classe 1\textsuperscript{a} Liceo Scientifico,
riguardanti le basi della cinematica, il moto rettilineo uniforme
e uniformemente accelerato.

Il percorso didattico è suddiviso in tre moduli, all'interno dei quali
sono affrontati i seguenti argomenti, ordinati secondo un criterio
di complessità crescente, partendo da un approccio completamente
qualitativo, per passare ad uno via via più quantitativo:
\begin{enumerate}
\item Posizioni, istanti, distanze e intervalli temporali, leggi orarie.
\item Velocità media, velocità istantanea, moto rettilineo uniforme.
\item Accelerazione media, accelerazione istantanea, moto uniformemente
      accelerato.
\end{enumerate}

Il percorso si fonda sul \emph{modello didattico delle 5E}\cite{bybee2006bscs}.
Ogni modulo si suddivide dunque in cinque sezioni, nelle quali sono presentate
attività, riflessioni, suggerimenti e problematiche, per ciascuna delle
cinque fasi di questo metodo di natura \emph{inquiry}:
\emph{engage}, \emph{explore}, \emph{explain}, \emph{extend}, \emph{evaluate}.

\section{Propedeuticità}
Al fine della buona riuscita di questo percorso, è necessario che gli
studenti coinvolti abbiano affrontato e consolidato la comprensione
dei seguenti argomenti:
\begin{itemize}
\item Operazioni algebriche elementari (es. somma, prodotto, frazioni
      e potenze).
\item Equazioni di primo grado.
\item Rappresentazioni di numeri su un asse ordinato.
\item Rappresentazione di coppie di numeri (punti) su un diagramma
      cartesiano.
\item \colorbox{red}{altro?}
\end{itemize}


\bibliography{bibliografia}{}
\bibliographystyle{plain}

\end{document}
