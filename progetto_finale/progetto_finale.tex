\documentclass{report} \usepackage[T1]{fontenc} \usepackage[italian]{babel}
\usepackage{color}
\usepackage[type={CC},modifier={by-sa},version={4.0},]{doclicense}
\usepackage{cite}

\usepackage{graphicx}
\graphicspath{ {./media/images/} }
\usepackage{float}
\usepackage{subcaption}

\usepackage{hyperref}

\usepackage{qrcode}
\usepackage{amsmath}

\addto\captionsitalian{\renewcommand{\bibname}{Riferimenti}}

\title{Title} \author{Daniele Melocchi\\Agnese Montanaro\\Matteo
Savatteri}

\begin{document}
\maketitle
\setcounter{page}{2}

Copyright Daniele Melocchi, Agnese Montanaro, Matteo Savatteri -
\the\year
\doclicenseThis

\vfill

Contatti:
\begin{itemize}
\item \textcolor{blue}{\href{mailto:daniele.melocchi@studenti.unimi.it}{daniele.melocchi@studenti.unimi.it}}
\item \textcolor{blue}{\href{mailto:agnese.montanaro@studenti.unimi.it}{agnese.montanaro@studenti.unimi.it}}
\item \textcolor{blue}{\href{mailto:matteo.savatteri@studenti.unimi.it}{matteo.savatteri@studenti.unimi.it}}
\end{itemize}

Il codice sorgente utilizzato per generare questo documento
è disponibile nella cartella \texttt{progetto\_finale} della
repository git raggiungibile presso l'indirizzo:
\\\\
\textcolor{blue}{\url{https://github.com/savaroskij/PED1}} \hfill \qrcode{https://github.com/savaroskij/PED1}

\thispagestyle{empty}

\tableofcontents

\chapter{Introduzione}
Questo documento presenta un ciclo di lezioni svolte in modalità
di didattica a distanza (\emph{DAD}) e
indirizzate ad una classe 1\textsuperscript{a} Liceo Scientifico,
riguardanti le basi della cinematica, il moto rettilineo uniforme
e uniformemente accelerato.

Il percorso didattico è suddiviso in tre moduli, all'interno dei quali
sono affrontati i seguenti argomenti, ordinati secondo un criterio
di complessità crescente, partendo da un approccio completamente
qualitativo, per passare ad uno via via più quantitativo:
\begin{enumerate}
\item Posizioni, istanti, distanze e intervalli temporali, leggi orarie.
\item Velocità media, velocità istantanea, moto rettilineo uniforme.
\item Accelerazione media, moto uniformemente
      accelerato.
\end{enumerate}

Il percorso si fonda sul \emph{modello didattico delle 5E}\cite{bybee2006bscs}.
Ogni modulo si suddivide dunque in cinque fasi, nelle quali sono presentate
attività, riflessioni, suggerimenti e problematiche affrontate, per ciascuna
di queste: \emph{engage}, \emph{explore}, \emph{explain}, \emph{extend},
\emph{evaluate}.

\section{Propedeuticità}
Al fine della buona riuscita di questo percorso, è necessario che gli
studenti coinvolti abbiano affrontato e consolidato la comprensione
dei seguenti argomenti:
\begin{itemize}
\item Operazioni algebriche elementari (es. frazioni e potenze).
\item Polinomi.
\item Equazioni di primo grado.
\item Rappresentazioni di numeri su un asse ordinato.
\item Rappresentazione di coppie di numeri (punti) su un diagramma
      cartesiano.
\item Conoscenza di grandezze fondamentali.
      (es. lunghezza, tempo)
\item Conoscenza di unità di misura di grandezze fondamentali.
      (es. metro, secondo)
\end{itemize}

\chapter{Posizioni, Istanti, Distanze, Intervalli Temporali} \label{posizioni_istanti}
Nel presente modulo lo studente familiarizza con i concetti
di posizione, distanza, istante (inteso come
\emph{lettura di orologio}) e intervallo di tempo.
Succesivamente, guidato dal docente, esplora le relazioni
che intercorrono tra queste nozioni nel contesto del moto di
un corpo, giungendo ad una comprensione qualitativa dei
concetti di \emph{evento} e \emph{legge oraria}.

\section{Engage}
\begin{itemize}
\item \textbf{Tempo richiesto in aula:} 15\textsuperscript{$\prime$}
\item \textbf{Materiale:} Computer
\end{itemize}

Il docente mostra agli studenti
\footnote{
Nel contesto della DAD, il docente può utilizzare
una piattaforma web di videotelefonia, che supporti la condivisione
di file multimediali. Jitsi Meet\textsuperscript{\textregistered},
Zoom\textsuperscript{\textregistered},
Google Meet\textsuperscript{\textregistered} e
Microsoft Teams\textsuperscript{\textregistered}
sono solo alcuni esempi.
}
il video di un fenomeno fisico riprodotto \emph{in reverse} (ovvero ribaltando
l'asse temporale). Il fenomeno fisico rappresentato dovrebbe essere scelto
in modo che sia difficile (o impossibile) distinguere se il video viene
riprodotto in reverse oppure no. Il moto di un pendolo semplice,
di un \emph{pendulum wave} o di altri moti periodici costituiscono buoni
esempi.

Nel contesto di questo progetto si è scelto il video in \emph{slow motion}
di un colibrì in volo, che si nutre da un tubicino (Figura \ref{fig:hummingbird})
\footnote{
\`E possibile scaricare il video presso
\textcolor{blue}{\href{https://github.com/savaroskij/PED1/blob/master/progetto_finale/media/video/Hummingbird.mp4?raw=true}{questa pagina web}}
. L'indirizzo del video originale si trova nei Riferimenti\cite{hbird}.
}.
Solamente la componente video, e non quella audio, è stata invertita per
aumentare l'effetto di inganno.
\begin{figure}
\centering
  \includegraphics[width=\textwidth]{Hummingbird}
  \caption{Un frame del video di un colibrì in volo che si nutre da un tubicino.}
  \label{fig:hummingbird}
\end{figure}
Il docente in seguito chiede agli studenti di descrivere quanto viene
visualizzato, e solamente infine svela che il video è riprodotto in
reverse. In questo modo l'insegnante ha l'occasione di far notare
allo studente, e lo esplicita, che per studiare qualsiasi fenomeno fisico
occorrono chiari riferimenti spaziali e temporali.

\section{Explore}
Questa fase viene svolta da ciascun studente a casa. Il professore propone agli
alunni tre attività distinte, attraverso le quali possano arrivare a
comprendere i concetti citati precedentemente (Inizio Capitolo \ref{posizioni_istanti}).

In primo luogo gli
studenti dovrebbero cogliere il significato di istante di tempo, o meglio
\emph{lettura di orologio}\cite{arons1997teaching}
\footnote{
           Il linguaggio utilizzato dal docente deve essere il più chiaro possibile, in modo
           che lo studente possa non fraintendere i concetti. Per esempio utilizzare il
           termine momenti, invece che letture di orologio, rimanda lo studente ad un
           concetto di tempo prolungato, non di istante.
         }
, e la differenza con intervallo
di tempo e in secondo luogo il significato di posizione e la differenza con distanza.
Tutte e tre le attività si basano sull’utilizzo della fotografia
\footnote{
          \`E opportuno che il tempo di esposizione per un singolo scatto sia ridotto, al
            fine di cogliere il concetto di istante temporale.
         }
e di grafici.

Emerge dalla letteratura, infatti, che l’utilizzo di grafici risulta essere un
ottimo strumento per la comprensione di concetti di base della cinematica
\cite{beichner1994testing}. I grafici possono essere
utili sia per descrivere il moto di un corpo osservato sia per prevedere il tipo
di moto a partire dal grafico stesso.

\subsection{Fotografare nel Tempo}

\begin{itemize}
\item \textbf{Materiale richiesto:} Smartphone, Fogli di Carta
\item \textbf{Tempo richiesto in aula:} 10\textsuperscript{$\prime$}
\end{itemize}

La prima attività proposta consiste nel fotografare diversi moti o situazioni
(minimo tre) scelti dallo studente in vari momenti (possono essere scattate sia
nella stessa giornata che in giorni differenti), quindi disegnare una linea del
tempo su un foglio e posizionare le foto scattate secondo i momenti scelti. Non
occorre scegliere una scala per la linea del tempo, perché interessa che lo
studente si accorga solo qualitativamente del fatto che possano passare periodi
di tempo di diversa durata tra eventi diversi. Inoltre in questa attività si
dovrebbe cogliere che il tempo ha una propria sequenzialità e che ciascuna
fotografia è un evento che accade in un istante determinato di tempo.

\`E opportuno che il docente guidi gli studenti mostrando loro un paio di esempi
di quanto richiesto, come una sequenza di foto di una pentola piena d’acqua
posta sul fuoco o episodi di vita quotidiana (Figura \ref{fig:asse_t_pentola}).
\begin{figure}[H]
\centering
  \includegraphics[width=\textwidth]{asse_t_pentola}
  \caption{Istanti del fenomeno di una pentola piena d'acqua posta sul fuoco,
           collocati sulla linea del tempo.}
  \label{fig:asse_t_pentola}
\end{figure}

\subsection{Fotografare nello Spazio}

\begin{itemize}
\item \textbf{Materiale richiesto:} Smartphone, Foglio di Carta
\item \textbf{Tempo richiesto in aula:} 15\textsuperscript{$\prime$}
\end{itemize}

Come seconda attività viene chiesto di scegliere un punto di partenza su un
corso o una via ed esplorarla camminando, fotografando palazzi o altri oggetti
incontrati (almeno 3 oggetti distinti). In seguito si disegni una linea su un
foglio e si incollino le diverse fotografie in base alla disposizione dei
soggetti immortalati lungo la via. Anche in questo caso non occorre che la
linea presenti una scala, in quanto l’obiettivo del lavoro è comprendere il
concetto di posizione di un corpo nello spazio e, in modo qualitativo, della
posizione relativa rispetto ad altri oggetti.

Anche in questo caso occorre che il professore mostri almeno uno più esempi
(Si veda la Figura \ref{fig:asse_s}).
\begin{figure}[H]
\centering
  \includegraphics[width=\textwidth]{asse_s}
  \caption{Vari oggetti immortalati lungo una via e rappresentati
           sulla linea della posizione.}
  \label{fig:asse_s}
\end{figure}

\subsection{Fotografarsi nel Tempo e nello Spazio}

\begin{itemize}
\item \textbf{Materiale richiesto:} Smartphone, Foglio di Carta
\item \textbf{Tempo richiesto in aula:} 10\textsuperscript{$\prime$}
\end{itemize}

Infine, la terza attività consiste nel ripercorrere (dall'inizio alla fine,
senza mai tornare indietro) la medesima via della
seconda esperienza, scattando dei \emph{selfie} e appuntandosi la lettura di orologio
corrispondente a ciascuna foto. Viene richiesta una elaborazione grafica in
analogia a quanto fatto nelle precedenti, senza che l’insegnante ne specifichi
la struttura finale.

Il professore per questa terza attività mostrerà solamente le fotografie
scattate, senza mostrare agli studenti il grafico ottenuto, in modo che gli
studenti possano ragionare su questo caso più complicato.
Gli studenti dovrebbero essere in grado di capire che per rappresentare
un corpo (loro stessi) che si muove nello spazio è necesserio utilizzare
due assi distinti, ovvero un diagramma cartesiano bidimensionale.
La Figura \ref{fig:piano_s_t} mostra un esempio del risultato atteso.

Tutte e tre le attività sono pensate per essere svolte incollando le fotografie
su un foglio, perché si ritiene che attraverso questa modalità operativa lo
studente  possa avere occasione di riflettere e giudicare i risultati ottenuti.
\begin{figure}[ht]
\centering
  \includegraphics[width=\textwidth]{piano_s_t}
  \caption{I \emph{selfie} della terza attività, rappresentati
           su un piano cartesiano.}
  \label{fig:piano_s_t}
\end{figure}

\section{Explain}\label{posizioni_istanti_explain}
\begin{itemize}
\item \textbf{Tempo richiesto:} 20\textsuperscript{$\prime$} di discussione tra
studenti + 30\textsuperscript{$\prime$} di dialogo con il docente
\end{itemize}

In questa fase viene permesso agli alunni di confrontarsi ed interpretare
quanto svolto nella fase precedente; il professore può intervenire per spiegare
o formalizzare alcuni concetti emersi.
L’insegnante \emph{in primis} suddivide la classe in gruppi di cinque studenti
\footnote{
          Per  esperienza degli autori, un numero esiguo o eccessivo di membri
          di un gruppo di lavoro rende difficile il coinvolgimento di tutti i
          partecipanti.
         }
e assegna
ciascun gruppo ad un’aula virtuale differente. La piattaforma utilizzata sarà a
discrezione della scuola
\footnote{
          Una possibilità è la piattaforma Zoom\textsuperscript{\textregistered},
          che  permette facilmente di creare aule virtuali di lavoro.
         }.
L’insegnante chiede di
discutere i grafici ottenuti nelle attività precedenti, confrontandoli, e
visita le aule virtuali per monitorare il lavoro ed eventualmente indirizzare
lo sguardo degli alunni con alcune domande. Per esempio: ``Che cosa potreste
dire della disposizione delle fotografie? Che significato potrebbe avere?''.
Ogni gruppo inizia quindi un \emph{brainstorming} e sceglie un portavoce,
il quale ha il  compito di esporre quanto emerso di fronte alla classe riunita
in un’unica aula virtuale con il professore.

Ciascuno studente nello svolgere la prima attività dovrebbe osservare che
alcune foto potrebbero essere più ravvicinate tra loro di altre
\ref{fig:asse_t_pentola}. Si ritiene che la maggioranza
interpreti la disomogeneità della disposizione delle
immagini come corrispondente a periodi di tempo di diversa durata tra i diversi
scatti. Il professore può formalizzare questo concetto utilizzando il termine
\emph{intervallo di tempo}, lasciando che gli studenti ne diano la definizione.
Sì osserverà che l’intervallo di tempo sarà dato dalla differenza tra le
letture di orologio delle due fotografie. Il docente a questo punto può
domandare se un singolo scatto occupi un intervallo di tempo. Si ritiene che la
maggior parte degli alunni abbia chiaro che la risposta sia no. La letteratura
fa intendere che un focus importante è assicurarsi che lo studente comprenda
che un istante di tempo equivale a \emph{zero} secondi\cite{arons1997teaching}
\footnote{
          Qualche alunno potrebbe argomentare il contrario, sostenendo che un singolo scatto
          occupi un intervallo di tempo molto breve. Questa ipotesi è corretta. Tuttavia,
          il tempo di un singolo scatto risulta essere su una scala di ordini di
          grandezza inferiore rispetto a quella utilizzata per l’attività. Per questo
          motivo si assume ad istante il tempo di una fotografia.
         }.
Ciascuna fotografia corrisponde ad un istante di tempo differente.

Dopo aver discusso i risultati della prima attività, risulta più immediato il
ragionamento sulla seconda. Un maggior numero di studenti è in grado di dire
che ciascuno oggetto fotografato è in un luogo preciso, cioè occupa una
posizione definita nello spazio. Alcuni possono costruire un’analogia tra
l’intervallo di tempo tra uno scatto e l’altro della prima attività e lo spazio
tra un oggetto e l’altro in questa esperienza. Si ipotizza che qualcuno in
classe riuscirà a far corrispondere l’aver camminato tra un oggetto incontrato
e l’altro con il concetto di \emph{distanza}.

Per quanto riguarda la terza esperienza è possibile che molti studenti si siano
trovati in difficoltà nella parte di rappresentazione grafica, in quanto meno
guidata delle precedenti. Tuttavia, è plausibile che in qualche gruppo di
lavoro sia emersa almeno l’impossibilità di utilizzare un’unica asse.
L’attività, infatti, comprendeva due grandezze differenti: tempo e spazio.
Nel caso nessun componente della classe sia arrivato ad ipotizzare un grafico
con due assi, che indicano rispettivamente spazio e tempo, allora il professore
è legittimato a guidarli in questa scoperta. Egli può far notare loro la
presenza di due informazioni distinte per ciascuna foto e invitarli ad un
paragone con i risultati delle prime due attività.
Inoltre in questa attività vi è un altro aspetto che si ritiene possa emergere
dalla discussione: il soggetto fotografato. Infatti, gli studenti potrebbero
osservare che, a differenza delle prime due attività, qui il soggetto è sempre
lo stesso in ogni scatto, cambiano solo i luoghi in cui si trova e le letture
di orologio. Dunque, il grafico caratterizzato dai due assi posizione-tempo
mostra dove si trova il soggetto, fissato un istante di tempo, ovvero rappresenta
il moto di questo corpo. Il professore sottolinea nuovamente l’esistenza di
due informazioni, la posizione e il tempo, e che esse siano in relazione tra
loro. Egli in particolare spiega che esse sono legate dal concetto
di \emph{legge oraria}:  una relazione che determina la posizione occupata da un
corpo  in moto (sempre il medesimo; in questa attività, sè stessi) ad un istante
fissato.

\section{Extend}
\begin{itemize}
\item \textbf{Tempo richiesto in aula:} 30\textsuperscript{$\prime$}
\item \textbf{Materiale:} Computer, Oggetti domestici
\end{itemize}

In questa fase si proprone un attività mirata ad estendere la consapevolezza
dello studente riguardo ad alcuni concetti presentati precedentemente.
In particolare si desidera estendere l'idea di posizione, che lo studente ha maturato,
allo spazio tridimensionale.

L'insegnante chiede a ciascuno studente di scegliere un oggetto nella stanza e di
spiegare alla classe, con le proprie parole, dove questo eggetto si trovi.
Lo studente potrebbe utilizzare altri oggetti nella stanza come riferimenti spaziali,
dicendo ad esempio ``L'oggetto si trova accanto alla finestra'', oppure
``L'oggetto si trova sopra il tavolo''; o forse potrebbe utilizzare sè stesso come
riferimento affermando ``L'oggetto si trova alla mia destra, in alto''.
A questo punto il docente chiede di scegliere altri due o tre oggetti,
e fa la stessa richiesta di localizzazione. Lo studente si renderà
conto dell'impossibilità di creare nella propria mente un'idea della stanza
degli altri compagni e della posizione di tutti gli oggetti, senza che
costoro esplicitino un riferimento spaziale univoco, e per ogni oggetto indichino tre distanze
misurate o stimate dal riferimento scelto.
L'insegnante dovrebbe guidare gli studenti in questo processo ponendo domande e fornendo suggerimenti
come: ``Dove si trova l'oggetto A rispetto all'oggetto B? E rispetto all'oggetto C?'',
oppure ``Dire che l'oggetto A si trova un metro sopra l'oggetto B è sufficiente
a far comprendere al tuo compagno la posizione dell'oggetto B all'interno della stanza?''
e ancora ``Forse potremmo dire che l'ogetto B si trova sopra l'oggetto A, ma anche alla sua destra
e qualche metro più avanti''.
Al termine di questo processo, il docente ufficializza la conoscenza acquisita disegnando
\footnote{
Il professore può disegnare su una lavagna e filmarsi durante l'operazione, oppure
utilizzare una tavoletta grafica o una lavagna virtuale (presente in software
come Zoom\textsuperscript{\textregistered}, ad esempio).
}
un diagramma cartesiano tridimensionale, collocando alcuni oggetti al suo interno
e osservando che la posizione dell'oggetto preso come riferimento per tutti gli altri
si chiama \emph{origine} e le tre distanze sono rappresentate da tre numeri lungo
gli assi, chiamati \emph{coordinate cartesiane} tridimensionali.

\section{Evaluate}
Per la valutazione di questo primo modulo del percorso il docente tiene conto della
partecipazione degli studenti alle varie attività proposte, del lavoro svolto sui
grafici richiesti e del loro coinvolgimento attivo all’interno dei gruppi di discussione.

\chapter{Velocità, Moto Rettilineo Uniforme}\label{velocità}
In questa fase lo studente verrà guidato alla comprensione del concetto di
velocità media e istantanea, e di legge oraria di un moto rettilineo
uniforme attraverso alcune attività. Occorre che il percorso sia guidato, perché
gli studenti di prima liceo affrontano per la prima volta questi argomenti e
il \emph{modus operandi} della fisica.

\section{Engage}\label{velocità_engage}

\begin{itemize}
\item \textbf{Tempo richiesto in aula:} 15\textsuperscript{$\prime$}
\item \textbf{Materiale:} Liquidi di Differente Viscosità, Contenitori Trasparenti Identici, Biglie
\end{itemize}

L’insegnante mostra agli studenti la caduta di una biglia in diversi liquidi. \`E
possibile eseguire questo esperimento sia in DAD, mostrando in telecamera i
contenitori con i liquidi, sia in classe portando tutto il materiale
occorrente. La scelta delle sostanze utilizzate è libera. L’unico criterio importante è
che i liquidi abbiano viscosità sensibilmente diverse, poiché in questo modo la velocità
della biglia all'interno del contenitore sarà visibilmete differente nei vari casi.

In questo progetto sono state scelte le seguenti sostanze: detersivo per i piatti,
acqua e olio di semi; posti in tre contenitori di pari altezza, trasparenti e di forma
il più possibile cilindrica (così da non avere effetti ottici di deformazione della
biglia che cade) (Figura \ref{fig:liquids}).
Le biglie utilizzate devono essere tutte uguali tra loro per massa e dimensione.
\begin{figure}[H]
\centering
  \begin{subfigure}[b]{0.5\textwidth}
  \includegraphics[width=\textwidth]{cat_caviar}
  \end{subfigure}
  \begin{subfigure}[b]{0.5\textwidth}
  \includegraphics[width=\textwidth]{cat_caviar}
  \end{subfigure}
  \begin{subfigure}[b]{0.5\textwidth}
  \includegraphics[width=\textwidth]{cat_caviar}
  \end{subfigure}
  \caption{
           I tre contenitori riempiti con i liquidi
           scelti per l'attività di engage sulla
           velocità: detersivo, acqua, olio di semi.
          }
  \label{fig:liquids}
\end{figure}
Prima di iniziare la fase operativa, il professore propone un gioco
agli studenti, chiedendo loro di indovinare in quale liquido la biglia
si muoverà più velocemente. \`E importante che in questa fase e per
tutto il resto del percorso didattico, il docente non utilizzi termini
che si riferiscono a concetti che gli studenti non hanno mai
affrontato prima. \`E necessario infatti che lo studente faccia
esperienza dei significati ai quali questi termini fanno riferimento,
prima che questi vengano introdotti. Questo vale in particolare
quando i termini in questione trovano un utilizzo in contesti
non scientifici\cite{arons1997teaching}.
Il professore dunque non parlerà mai di \emph{velocità} durante
la presentazione di questa attività, ma dirà ad esempio:
``Riuscite a prevedere quale biglia percorrerà la \emph{distanza}
dalla sommità alla base del recipiente nell'\emph{intervallo di tempo}
minore?'', utilizzando i termini introdotti nella Sezione
\ref{posizioni_istanti_explain}.

In seguito l'insegnante effettua l'esperimento davanti alla classe.
Se l'esperienza viene mostrata in presenza, il professore
può chiedere a tre studenti di lasciar cadere la biglia nei tre recipienti
simultaneamente; in questo modo gli alunni saranno facilitati
nell'osservare quale biglia giunge prima e quale dopo sul fondo del
recipiente. Nel caso della DAD, le cadute verranno mostrate in successione,
 ma dovrà avere particolare cura di scegliere tre sostanze dalla
viscosità molto diversa
\footnote{
          \`E possibile scaricare il video della biglia che
          cade nel detersivo presso
          \textcolor{blue}{\href{https://github.com/savaroskij/PED1/blob/master/progetto_finale/media/video/biglia_detersivo.mp4?raw=true}{questo indirizzo web}}.
          Lo stesso esperimento con acqua si può trovare
          \textcolor{blue}{\href{http://burymewithmymoney.com/}{qui}},
          e
          \textcolor{blue}{\href{http://burymewithmymoney.com/}{qui}}
          quello con olio di semi.
         }.

\section{Explore}
In questa fase il professore propone agli alunni diverse attività da svolgere a
casa per arrivare a comprendere i concetti citati ad inizio Capitolo
\ref{velocità}.

\section{Dal Grafico al Moto}

\begin{itemize}
\item \textbf{Tempo richiesto in aula:} 15\textsuperscript{$\prime$}
\item \textbf{Materiale:} Nastro adesivo, Foglio di Carta
\end{itemize}

Per la prima attività l’insegnante mostra agli studenti un piano cartesiano su
cui sono posizionati alcuni punti e i cui assi rappresentano spostamento e
tempo (Vedi Figura \ref{fig:kine_plot1}).
\begin{figure}[ht]
\centering
  \includegraphics[width=\textwidth]{kine_plot1}
  \caption{Grafico con punti per la prima attività
           cinestetica.}
  \label{fig:kine_plot1}
\end{figure}
Successivamente egli inquadra con la telecamera la sua mano
poggiata sul bordo del tavolo di lavoro (Come in  Figura \ref{fig:kine_hand}),
\begin{figure}[H]
\centering
  \includegraphics[width=\textwidth]{kine_hand}
  \caption{Una mano che si muove sul bordo del tavolo durante la
           prima attività cinestetica.}
  \label{fig:kine_hand}
\end{figure}
dove sono stati precedentemente incollati dei pezzi di nastro adesivo
equidistanti tra loro (non è rilevante la lunghezza della distanza
tra le strisce di nastro, e può essere scelta arbitrariamente). A questo
punto il docente muove la mano secondo quanto
si legge nel grafico in Figura \ref{fig:kine_plot1}:
\footnote{\`E possibile scaricare i video dell'attività cinestetica presso
            \textcolor{blue}{\href{https://github.com/savaroskij/PED1/tree/master/progetto_finale/media/video/kine_videos_1}{questo indirizzo web}}.
            Si veda per questo particolare esempio il file \texttt{vid\_plot\_1\_scale.mp4}.
         }
sia le posizioni occupate dalla
mano che i tempi tra un movimento e l’altro devono qualitativamente
rispettare le informazioni presenti nel grafico. I riferimenti posti sul tavolo
aiutano a mantenere le corrette proporzioni. Inoltre, potrebbe essere utile un
dispositivo che aiuti a scandire il tempo, quale per esempio un metronomo. Nel
caso in cui sia possibile la didattica in presenza, l’attività verrà mostrata
disegnando il grafico alla lavagna e poggiando la mano sul bordo della
cattedra. Il professore mostra la medesima attività, usando un paio di grafici
più complessi, in cui non vi sono più solo dei punti, ma delle curve (Vedi Figura
\ref{fig:kine_plot3}).
Viene chiesto agli studenti di ripetere a casa l’attività usando
nuovi grafici analoghi (Si veda l'Appendice \ref{appendix_kine}) e riprendendo
per ciascuno la propria mano in movimento.
Nella consegna viene richiesto anche
di scegliere uno dei grafici e di descrivere per iscritto il moto osservato della
propria mano.

\begin{figure}[H]
\centering
  \includegraphics[width=\textwidth]{kine_plot2}
  \label{fig:kine_plot2}
\end{figure}

\begin{figure}[H]
\centering
  \includegraphics[width=\textwidth]{kine_plot3}
  \caption{Grafici con spezzate per la prima attività
           cinestetica.}
  \label{fig:kine_plot3}
\end{figure}


\section{Dal Moto al Grafico}

\begin{itemize}
\item \textbf{Tempo richiesto in aula:} 15\textsuperscript{$\prime$}
\item \textbf{Materiale:} Nastro adesivo, Foglio di Carta
\end{itemize}

La seconda attività proposta agli studenti consiste nello svolgere il
procedimento opposto rispetto all’attività precedente. Il professore presenta
agli studenti dei video che mostrano  alcuni moti della sua mano lungo il bordo
del tavolo (un minimo di tre) e chiede loro di graficarli. In questo modo
ciascuno studente può concentrarsi sul moto del proprio corpo, comprendendo più
approfonditamente la relazione tra il moto della la mano e cosa viene riportato
sul grafico
\footnote{
          Esempi di moti sono mostrati nei video presso
          \textcolor{blue}{\href{https://github.com/savaroskij/PED1/tree/master/progetto_finale/media/video/kine_videos_2}{questa pagina web}}.
}
In caso fosse possibile lavorare in classe, il moto rappresentato sui grafici
può essere riprodotto dagli studenti con il proprio corpo, camminando in linea
retta. Utilizzando un sensore di prossimità, è possibile controllare in
tempo reale i movimenti degli studenti e appurare che il loro moto sia
attinente a quello riportato nel grafico.

\section{Velocità e Moto Rettilineo Uniforme}

\begin{itemize}
\item \textbf{Tempo richiesto in aula:} 30\textsuperscript{$\prime$}
\item \textbf{Materiale:} Biglia, Contenitore Trasparente, Liquido, Cronometro, Righello
\end{itemize}

Il docente propone un’ultima attività che permetta agli studenti di osservare
ed indagare un esempio di moto rettilineo uniforme. Una possibilità potrebbe
essere quella di lasciar cadere una biglia all’interno di un contenitore
trasparente riempito con una sostanza liquida (analogo a quello utilizzato
nella Sezione \ref{velocità_engage}, si veda Figura \ref{fig:liquids}).
Occorre che l’insegnante abbia
le stesse accortezze discusse nella Sezione \ref{velocità_engage} e
le espliciti ai suoi studenti. Inoltre si ritiene di fondamentale
importanza la scelta del liquido, per questo
l’insegnante deve verificare prima quale sia il più appropriato da suggerire
agli studenti. Il criterio di scelta si basa sulla viscosità del liquido, la
quale deve essere sufficiente affinché la biglia raggiunga una velocità
prossima a quella limite molto rapidamente, proseguendo poi di moto uniforme
(in prima approssimazione)
lungo il resto del contenitore. Un esempio di sostanza potrebbe essere il
detersivo per i piatti. Agli alunni viene dunque chiesto di suddividere l'altezza
del contenitore in intervalli spaziali uguali tra loro (per esempio marcandoli con
un pennarello), e di misurare questi ultimi e i tempi necessari
alla biglia per percorrere ciascun
tratto \footnote{\`E possibile scaricare il video dell'esperimento presso
          \textcolor{blue}{\href{https://github.com/savaroskij/PED1/blob/master/progetto_finale/media/video/biglia_detersivo.mp4?raw=true}{questo indirizzo web}}.
         }.
Il docente chiede agli studenti di riportate in una tabella i dati
misurati e di rappresentare il moto della biglia in un diagramma
spazio tempo, esplicitando che per compiere quest'ultima operazione
sarà necessario sommare di volta in volta tutte le distanze percorse
dalla biglia fino alla posizione di interesse. Lo stesso vale per
gli intervalli di tempo.
Questa attività è
spendibile sia in DAD che in classe. In questo secondo scenario gli studenti
vengono divisi in gruppi di tre per svolgere l’esperimento.

\section{Explain}\label{velocità_explain}

\begin{itemize}
\item \textbf{Tempo richiesto in aula:} 40\textsuperscript{$\prime$}
      + 60\textsuperscript{$\prime$} di discussione con il professore
\end{itemize}

Dopo che gli studenti hanno svolto tutte le attività proposte, occorre un
confronto con l’insegnante e con i propri compagni di classe su quanto
osservato. A questo scopo la classe viene divisa in gruppi di cinque, nei quali gli
studenti hanno il tempo di discutere e confrontare i risultati degli esercizi
svolti. Questo primo confronto coinvolge le prime due attività svolte. Dopo
circa mezz’ora l’insegnante riunisce tutti gli alunni e chiede ad un
rappresentante di ciascun gruppo di presentare quanto emerso.
Si presume che la maggior parte degli studenti riescano ad interpretare i
grafici in relazione al movimento della mano: due spostamenti uguali percorsi
in intervalli di tempo di durata diversa corrispondono ad un movimento più
rapido se il periodo di tempo è minore e più lento se è maggiore; osservazioni
analoghe possono essere fatte fissando l’intervallo di tempo e osservando
spostamenti diversi.
A questo punto dovrebbe essere chiaro agli studenti che le grandezze in gioco
-che influenzano la rapidità del movimento della mano- sono la distanza compiuta
e l’intervallo di tempo impiegato per compierla. Viene chiesto
agli studenti di provare ad ipotizzare quale potrebbe essere l’espressione
matematica che meglio descrive questa relazione. Una possibile ipotesi è che
essa sia data dal prodotto tra la distanza e il tempo impiegato per
percorrerla. Il docente tuttavia può far notare che in tal caso non sarebbe
possibile che un uguale aumento della distanza e del tempo impiegato non portino
variazione nella rapidità, mentre è possibile osservarlo nei grafici. Si
ritiene, invece, che pochi alunni possano ipotizzare che la relazione cercata
sia data dalla somma o dalla differenza. Nel caso accada, il professore guida i
ragazzi nel comprendere che non ha senso  sommare grandezze fisiche diverse
(si può osservare anche dalle unità di misura). Infine, si
ritiene che qualche studente giunga alla risposta corretta, ovvero che la
relazione sia data dal rapporto tra distanza e intervallo di tempo impiegato.
Questo risulta essere ragionevole, anche perché il rapporto è l'operazione
di confronto tra due quantità.
Grazie al lavoro fatto nel Capitolo \ref{posizioni_istanti}, gli studenti sanno
che sia la  distanza che l’intervallo di tempo sono dati da una differenza
di posizione nel  primo caso e di tempo nel secondo. Allora l’espressione
che si ottiene è la seguente:

\begin{equation}
\textrm{velocità media} = \frac{\textrm{distanza}}{\textrm{intervallo di tempo}} =
\frac{s_f - s_i}{t_f - t_i}
\end{equation}

Il professore riporta all’attenzione degli studenti i grafici delle spezzate e
chiede loro quale sia la differenza tra ciascun tratto. Alcuni studenti saranno
in grado di rispondere che si tratta della pendenza, altri forse saranno più
specifici e useranno il termine coefficiente angolare. Si ritiene possibile che
qualche studente noti che l’espressione della velocità media appena incontrata
corrisponda al coefficiente angolare di ciascun segmento; Altrimenti
l’insegnante li guiderà in questo confronto.
Dopo aver analizzato in profondità le prime due attività e aver dedotto il
concetto di velocità media, si lavora sull’ultima attività proposta: la caduta
di una biglia nel detersivo. Anche in questo caso si lasciano ai gruppi venti minuti
per confrontarsi sui dati raccolti
\footnote{
          I risultati dell'analisi dati di questo esperimento, svolto
          nell'ambito di questo progetto, sono presentati nell'Appendice
          \ref{appendix_data_analysis}.
         }.
Successivamente si
inizia un dialogo con il docente, riportando quanto osservato. La maggior parte
dei gruppi dovrebbe concludere che la biglia impiega tempi simili a percorrere
ciascun tratto, ovvero che tratti uguali vengono percorsi in tempi uguali.
Grazie al lavoro precedente gli studenti possono affermare che la velocità
media risulta essere uguale per ciascun tratto. Questa ipotesi è supportata sia
dal fatto che il grafico spostamento-tempo mostra che tutti i segmenti disegnati
hanno la stessa pendenza,
sia dal fatto che ora gli studenti sono in grado di calcolare la velocità
media per ciascuna coppia distanza-intervallo di tempo.
L’insegnante fa notare che le velocità medie sono uguali nei
diversi tratti, nonostante il numero di tratti scelto possa variare da studente
a studente. Questo può portare ad ipotizzare che tale osservazione rimarrebbe
vera anche con un numero elevato di tratti. Il docente allora è giustificato
nel definire questo tipo di moto \emph{uniforme}
\footnote{
          In effetti un moto uniforme è
          tale se la velocità è costante in ogni istante di tempo. Occorrerebbero,
          dunque, infiniti tratti. Ma l’approssimazione di un numero elevato di tratti è
          sufficiente per quanto discusso in questa fase, dove ancora non si è affrontato
          il concetto di velocità istantanea.
         }.
Infine, è importante far osservare agli
alunni che la biglia cade verticalmente. Si potrebbe affermare che tracci una
retta verticale. Dunque il moto è definito come \emph{rettilineo uniforme}. Questa
fase si conclude con la scoperta della legge oraria di tale moto. Il professore
chiede ai ragazzi di trovare l’espressione della velocità media per ciascun
tratto percorso dalla biglia. Successivamente li guida nell’invertire ciascuna
espressione così da ottenere i relativi spostamenti:

\begin{align}
\Delta s_1 &= v_m \Delta t_1 \\
\Delta s_2 &= v_m \Delta t_2 \\
\Delta s_3 &= v_m \Delta t_3 \\
\Delta s_4 &= v_m \Delta t_4
\end{align}

L’insegnante utilizza le espressioni degli spostamenti di ciascun tratto
per far notare che differiscono solo per l’indice, che indica il tratto di
riferimento, mentre la velocità $v_m$ è sempre la stessa. \`E dunque possibile
giungere ad una espressione generale: $\Delta s= v \Delta t$, dove $\Delta s$
è la differenza tra posizione finale e posizione iniziale.
Infine si ottiene la legge oraria del moto rettilineo uniforme esplicitando la
posizione finale:

\begin{equation}
s = s_0 + v (t - t_0)
\end{equation}

\pagebreak

\section{Extend}

\begin{itemize}
\item \textbf{Tempo richiesto in aula:} 30\textsuperscript{$\prime$}
\item \textbf{Materiale:} Computer
\end{itemize}

In questa fase si propone che lo studente estenda la propria conoscenza
riguardo la velocità, introducendo l'idea di \emph{velocità istantanea}.
Uno studente di 1\textsuperscript{a} liceo scientifico non dispone
degli strumenti matematici necessari per comprendere la definizione
di velocità istantanea, che richiederebbe il concetto
di \emph{limite} e di \emph{derivata}. Egli tuttavia, sulla base
dei concetti introdotti nelle fasi precedenti, può acquisire
una conoscenza intuitiva dell'operazione di limite, che permette
il passaggio dalla velocità media alla velocità istantanea.

Il docente per aiutare lo studente a comprendere questa idea,
mostra un video di un corpo in caduta libera. Per questo progetto
si è scelto un estratto del lancio di Felix Baumgartner da un
pallone aerostatico nella stratosfera
\footnote{
          \`E possibile scaricare l'estratto del video presso
          \textcolor{blue}{\href{https://github.com/savaroskij/PED1/blob/master/progetto_finale/media/video/felix_baumgartner.mp4?raw=true}{questo indirizzo web}}.
          L'indirizzo del video originale si trova nei Riferimenti\cite{felix}.
         }.
L'insegnante chiede agli studenti di descrivere cosa accade nel
video, concentrandosi in particolare sulla velocità di Felix in caduta
libera. Gli studenti potranno osservare che dopo il lancio la velocità
dell'astronauta aumenta rapidamente e dunque non rimane costante come nel
caso delle biglia che cade in un fluido molto viscoso.
Gli studenti potrebbero utilizzare frasi come ``L'astronuata cade
sempre più velocemente'' oppure ``L'astronauta parte
lentamente e poi va più veloce''.
Il professore dovrebbe porre l'attenzione della classe sulle
risposte più simili alla prima affermazione. Quest'ultima
infatti evidenzia che \emph{in ogni istante} la velocità
dell'astronauta in caduta sta cambiando (e nel caso particolare aumenta).
Il professore aiuta gli studenti a giungere ad una comprensione
più lucida di questa realizzazione, utilizzando il software
\emph{GeoGebra}
\footnote{
          GeoGebra è una suite di programmi per la didattica della matematica
          disponibile per iOS, Android, GNU/Linux, Windows, Mac OS, Chromebook.
          Per maggiori informazioni, si visiti
          \textcolor{blue}{\href{https://www.geogebra.org/}{questo sito web}}.
         }.
Egli disegna con il programma un ramo di parabola con concavità
verso il basso e lo mostra agli studenti (Si veda la Figura
\ref{fig:geogebra}), chiedendo di immaginare che questo grafico
rappresenti il moto di Felix nella stratosfera.
\begin{figure}[H]
\centering
  \begin{subfigure}[b]{0.49\textwidth}
  \includegraphics[width=\textwidth]{tg1}
  \end{subfigure}
  \begin{subfigure}[b]{0.49\textwidth}
  \includegraphics[width=\textwidth]{tg2}
  \end{subfigure}
  \begin{subfigure}[b]{0.49\textwidth}
  \includegraphics[width=\textwidth]{tg3}
  \end{subfigure}
  \begin{subfigure}[b]{0.49\textwidth}
  \includegraphics[width=\textwidth]{tg4}
  \end{subfigure}
  \caption{
           Rami di parabola  disegnati con GeoGebra.
          }
  \label{fig:geogebra}
\end{figure}
La scala sugli assi non è importante e neppure che gli studenti sappiano quale
curva è rappresentata nel grafico.
Il professore chiede agli studenti di concentrarsi su un tratto arbitrario della
caduta di Felix e di figurarlo nella propria mente; poi sceglie due punti
della curva, li rappresenta grazie al software e chiede agli studenti
come si potrebbe rappresentare sul grafico la velocità media dell'astronauta
in quel preciso tratto, compreso tra i punti A e B.
Si suppone che gli studenti, in virtù di quanto appreso
nella Sezione \ref{velocità_explain}, sapranno indicare al professore di
disegnare una retta che congiunga i due punti presi in considerazione.
Il docente può aiutare gli studenti rappresentando nel grafico le proiezioni
dei punti scelti sugli assi ordinati.
L'insegnante rappresenta la retta secante sul diagramma cartesiano e
successivamente, utilizzando lo \emph{slider} presente nel programma,
avvicina il punto B al punto A. Gli studenti osservano che l'inclinazione
della secante è cambiata e questa si avvicina sempre più alla curva.
Il professore ripete l'operazione altre due o
tre volte, avvicinando sempre di più i punti A e B (permettendo ai ragazzi
ogni volta di osservare la nuova retta secante) fino a che i due punti
coincidono (Come si vede nella Figura \ref{fig:geogebra}).

Il professore suggerisce agli studenti che ora la retta rappresentata nel grafico
è la retta tangente alla curva nel punto A, e chiede agli studenti se questa possa
ancora rappresentare una velocità media. Gli studenti, guidati al caso limite dalla retta
secante alla retta tangente, hanno ora la possibilità di realizzare che il coefficiente
angolare di quest'ultima rappresenta la velocità media corrispondente ad un intervallo
di tempo di durata \emph{zero} secondi, ovvero la ``\emph{velocità media di un istante}''.
Il docente ufficializza la conoscenza acquisita, spiegando che questa grandezza è
chiamata \emph{velocità instantanea} o più semplicemente \emph{velocità}.

\section{Evaluate}
Per la valutazione di questo secondo modulo del percorso il docente tiene conto della
partecipazione degli studenti alle varie attività proposte a casa e in classe,
del lavoro svolto sui grafici e i video richiesti, e del loro coinvolgimento attivo
all’interno dei gruppi di discussione.

\chapter{Accelerazione, Moto Rettilineo Uniformemente Accelerato}
Nel presente modulo lo studente familiarizza con il concetto di accelerazione
media e con la legge oraria del moto uniformemente accelerato.
Infine egli giunge ad una comprensione più profonda della differenza
tra accelerazione e velocità.

\section{Engage}

\begin{itemize}
\item \textbf{Tempo richiesto in aula:} 15\textsuperscript{$\prime$}
\item \textbf{Materiale:} Computer
\end{itemize}

Il professore mostra agli studenti dei video interessanti di moti uniformemente
accelerati. Nel contesto di questo progetto si suggerisce di riprodurre un
filmato in cui viene mostrata la caduta di un martello e di
una piuma in ambiente lunare (Figura \ref{fig:hammer_feather_moon})
\footnote{
          \`E possibile scaricare i video utilizzati in questo engage:
            \textcolor{blue}{\href{https://github.com/savaroskij/PED1/blob/master/progetto_finale/media/video/Hammer-Feather_Moon.mp4?raw=true}{indirizzo primo video}},
            \textcolor{blue}{\href{https://github.com/savaroskij/PED1/blob/master/progetto_finale/media/video/Ball-Feather_Vacuum.mp4?raw=true}{indirizzo secondo video}}.
            I link ai video originali si trovano nei
            riferimenti\cite{moon}\cite{bowling}.
         }.
\begin{figure}[H]
\centering
  \includegraphics[width=\textwidth]{hammer_feather_moon}
  \caption{Un fotogramma del video di un astronauta sulla Luna che lascia
           cadere un martello e una piuma.}
  \label{fig:hammer_feather_moon}
\end{figure}
Il docente interrompe il filmato nell’istante
precedente alla caduta e domanda agli studenti che cosa si aspettano accada. Si
presume che gli studenti non prevedano che gli oggetti arrivino al suolo
insieme.
Un secondo filmato in slow motion riproduce la stessa esperienza nell’ambiente
controllato di un laboratorio, permettendo agli studenti di poter apprezzare
maggiormente il moto descritto.

Questa attività cattura l’attenzione degli studenti e consente al docente
di dirigere la curiosità dei propri alunni sui moti accelerati.

\section{Explore}

\section{Explain}\label{a_explain}

\section{Extend}
\begin{itemize}
\item \textbf{Tempo richiesto in aula:} 60\textsuperscript{$\prime$}
\item \textbf{Materiale:} Smartphone, Computer, Oggetti domestici
\end{itemize}

In questa fase si desidera che lo studente estenda la sua consapevolezza
riguardo il concetto di accelerazione. La letteratura documenta una difficoltà
in una frazione significativa degli studenti nel distinguere tra velocità
e accelerazione. Tale difficoltà si manifesta in particolare nello studio
del fenomeno conosciuto come \emph{Top of the Flight}, o in generale
in tutti i casi nei quali si verifica che la velocità istantanea sia nulla
e contemporaneamente l'accelerazione non lo sia\cite{arons1997teaching}
\cite{trowbridge1981investigation}.

Gli studenti hanno già fatto esperienza di questa situazione nella Sezione
\ref{a_explain}, durante l'analisi dell'esperimento della bicicletta che
scende da una rampa. Quando la bicicletta si trova ferma in cima alla rampa
e inizia la sua discesa, per un solo istante (quello iniziale) la sua velocità
è nulla, mentre la sua accelerazione ha un valore non nullo, che rimmarrà costante
lungo tutta la discesa.

L'insegnante presenta alla classe un surrogato in scala dell'esperimento della
bicicletta, perché possa essere studiato con l'applicazione per smartphone
\emph{Phyphox}
\footnote{
          Phyphox è un applicazione \emph{Free Software} per dispositivi mobili
          (Sistemi supportati Android e IOS) che trasforma il proprio
          smartphone in un piccolo laboratorio di fisica. Maggiori
          informazioni e documentazione sono reperibili presso
          \textcolor{blue}{\href{https://phyphox.org/}{questa pagina web}}.
         }.
In principio il docente presenta agli studenti l'app, mostrando loro dove
reperirla e le sue funzioni, in particolare lo strumento di misura
\emph{Acceleration (Whitout g)} (Figura \ref{fig:phyphox}) e la funzione
\emph{Save experiment state}.
L'insegnante dovrebbe mostrare un rapido ma efficave esempio pratico
di utilizzo di questo strumento, per esempio attivando la funzione
di registrazione dati, lanciando in aria il telefono, riprendendolo
al volo e mostrando il risultato agli studenti.
\begin{figure}[H]
\centering
  \includegraphics[width=\textwidth]{phyphox}
  \caption{Uno \emph{screenshot} della schermata dello strumento
           \emph{Acceleration (Whitout g)}
           nell'applicazione Phyphox.}
  \label{fig:phyphox}
\end{figure}
In seguito il professore assegna agli studenti un compito da eseguire
a casa: indica loro di scegliere una superfice inclinata
e priva di attrito
\footnote{
          Si ricorda che gli studenti di una classe prima liceo
          non hanno familiarità con il concetto di attrito,
          dunque il professore dovrebbe utilizzare termini differenti
          come ``Superficie inclinata e liscia''.
         },
e di far scivolare il telefono lungo la superficie, attivando
la registrazione dati dello strumento appena mostrato.
Il compito di scegliere l'inclinazione del piano, il modo in cui
collocare il telefono sopra di esso e altre osservazioni pratiche
sono lasciate allo studente.

Durante la lezione successiva l'insegnate chiede agli studenti
di condividere la propria esperienza e di spiegare l'interpretazione
che questi danno ai grafici prodotti dall'applicazione.
Il docente dovrebbe aver riprodotto l'esperienza, in modo
da poter presentare agli studenti i propri risultati in una
forma più adatta alla comprensione degli stessi.
Si ritiene che i grafici prodotti da Phyphox siano utili,
in quanto di accesso immediato, ma di difficile interpretazione
per uno studente. Il docente potrà esportare i propri
dati nel formato preferito grazie alla funzione
\emph{Export Data}, e in seguito analizzarli con uno
strumento di propria scelta.
Una possibile analisi è mostrata nella Figura \ref{fig:a_phyphox}.
\begin{figure}[H]
\centering
  \includegraphics[width=\textwidth]{a_phyphox}
  \caption{Il plot, realizzato con il modulo \emph{Matplotlib}
           del linguaggio di programmazione \emph{Python},
           mostra l'accelerazione lungo il piano misurata da
           Phyphox. I dati sono stati mediati nel tempo
           per rimuovere il rumore e un fit lineare è stato
           eseguito sulla coda.
          }
  \label{fig:a_phyphox}
\end{figure}
Il professore aiuta gli studenti ad analizzare il grafico ponendo
alcune domande: ``Riuscite ad individuare nel grafico l'istante
in cui il telefono è stato lasciato libero di scivolare?'',
``Quanto vale l'accelerazione prima che il telefono inizi a
scivolare?'', ``Quanto vale l'accelerazione nell'istante
in cui il telefono viene lasciato scivolare, ma è ancora fermo,
ovvero la sua velocità vale zero? E mentre il telefono sta
scivolando?''. In questo modo egli guida gli studenti verso
la realizzazione che un corpo avente velocità nulla ad un certo
istante, non necessariamente deve avere anche un'accelerazione nulla.

\appendix
\chapter{Grafici per le Attività Cinestetiche}\label{appendix_kine}
Sono qui presentati alcuni esempi di grafici che il docente può
utilizzare per le attività cinestetiche.

\begin{figure}[H]
\centering
  \includegraphics[width=\textwidth]{kine_plot4}
  \label{fig:kine_plot4}
\end{figure}

\begin{figure}[H]
\centering
  \includegraphics[width=\textwidth]{kine_plot5}
  \caption{Grafici di curve per le attività
           cinestetiche.}
  \label{fig:kine_plot5}
\end{figure}

\chapter{Analisi Dati}\label{appendix_data_analysis}
\section{Moto Rettilineo Uniforme}

\begin{figure}[H]
\centering
  \includegraphics[width=\textwidth]{fit_marble}
  \caption{Fit lineare dei dati per l'esperimento della biglia che cade nel detersivo,
           un esempio di moto rettilineo uniforme.}
  \label{fig:fit_marble}
\end{figure}

\begin{table}[H]
  \renewcommand{\arraystretch}{1.5}
  \centering
  \begin{tabular}{ | c | }
    \hline
    $v$ [$m/s$] \\
    \hline
    $0.01180\pm0.00010$ \\
    \hline
  \end{tabular}
  \caption{Valore atteso ed errore della velocità da fit con il metodo dei minimi
           quadrati.}
  \label{tab:fit_marble}
\end{table}

\section{Moto Rettilineo Uniformemente Accelerato}

\begin{figure}[H]
  \centering
  \includegraphics[width=\textwidth]{fit_bike}
  \caption{Fit dei dati per l'esperimento della bicicletta che scende da una rampa,
           un esempio di moto rettilineo uniformemente accelerato.}
  \label{fig:fit_bike}
\end{figure}

\begin{table}[H]
  \renewcommand{\arraystretch}{1.5}
  \centering
  \begin{tabular}{ | c | c | c | }
    \hline
    $a$ [$m/s^2$] &  $\theta$ [deg] & pendendenza [\%] \\
    \hline
    $0.706\pm0.015$ & $4.13\pm0.09$ & $7.21\pm0.15$ \\
    \hline
  \end{tabular}
  \caption{Valore atteso ed errore dell'accelerazione, dell'angolo di inclinazione della rampa
           e della pendenza percentuale da fit con il metodo dei minimi quadrati.}
  \label{tab:fit_bike}
\end{table}

\bibliography{bibliografia}{}
\bibliographystyle{plain}
\addcontentsline{toc}{chapter}{Riferimenti}

\end{document}
