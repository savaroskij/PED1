\documentclass{report} \usepackage[T1]{fontenc} \usepackage[italian]{babel}
\usepackage{color}
\usepackage[type={CC},modifier={by-sa},version={4.0},]{doclicense}
\usepackage{cite}

\usepackage{graphicx}
\graphicspath{ {./media/images/} }
\usepackage{float}

\usepackage{hyperref}

\addto\captionsitalian{\renewcommand{\bibname}{Riferimenti}}

\title{Title} \author{Daniele Melocchi\\Agnese Montanaro\\Matteo
Savatteri}

\begin{document}
\maketitle
\setcounter{page}{2}

Copyright Daniele Melocchi, Agnese Montanaro, Matteo Savatteri -
\the\year \doclicenseThis \thispagestyle{empty}

\tableofcontents

\chapter{Introduzione}
Questo documento presenta un ciclo di lezioni svolte in modalità
di didattica a distanza (\emph{DAD}) e
indirizzate ad una classe 1\textsuperscript{a} Liceo Scientifico,
riguardanti le basi della cinematica, il moto rettilineo uniforme
e uniformemente accelerato.

Il percorso didattico è suddiviso in tre moduli, all'interno dei quali
sono affrontati i seguenti argomenti, ordinati secondo un criterio
di complessità crescente, partendo da un approccio completamente
qualitativo, per passare ad uno via via più quantitativo:
\begin{enumerate}
\item Posizioni, istanti, distanze e intervalli temporali, leggi orarie.
\item Velocità media, velocità istantanea, moto rettilineo uniforme.
\item Accelerazione media, accelerazione istantanea, moto uniformemente
      accelerato.
\end{enumerate}

Il percorso si fonda sul \emph{modello didattico delle 5E}\cite{bybee2006bscs}.
Ogni modulo si suddivide dunque in cinque fasi, nelle quali sono presentate
attività, riflessioni, suggerimenti e problematiche affrontate, per ciascuna
di queste: \emph{engage}, \emph{explore}, \emph{explain}, \emph{extend},
\emph{evaluate}.

\section{Propedeuticità}
Al fine della buona riuscita di questo percorso, è necessario che gli
studenti coinvolti abbiano affrontato e consolidato la comprensione
dei seguenti argomenti:
\begin{itemize}
\item Operazioni algebriche elementari (es. frazioni e potenze).
\item Polinomi.
\item Equazioni di primo grado.
\item Rappresentazioni di numeri su un asse ordinato.
\item Rappresentazione di coppie di numeri (punti) su un diagramma
      cartesiano.
\item Conoscenza di grandezze fondamentali.
      (es. lunghezza, tempo)
\item Conoscenza di unità di misura di grandezze fondamentali.
      (es. metro, secondo)
\end{itemize}

\chapter{Posizioni, Istanti e Intervalli}
Nel presente modulo lo studente familiarizza con i concetti
di posizione, distanza, istante (inteso come
\emph{lettura di orologio}) e intervallo di tempo.
Succesivamente, guidato dal docente, esplora le relazioni
che intercorrono tra queste nozioni nel contesto del moto di
un corpo, giungendo ad una comprensione qualitativa dei
concetti di \emph{evento} e \emph{legge oraria}.

\section{Engage}
\begin{itemize}
\item \textbf{Tempo richiesto in aula:} 15\textsuperscript{$\prime$}
\item \textbf{Materiale:} Computer
\end{itemize}

Il docente mostra agli studenti
\footnote{
Nel contesto della DAD, il docente può utilizzare
una piattaforma web di videotelefonia, che supporti la condivisione
di file multimediali. Jitsi Meet\textsuperscript{\textregistered},
Zoom\textsuperscript{\textregistered},
Google Meet\textsuperscript{\textregistered} e
Microsoft Teams\textsuperscript{\textregistered}
sono solo alcuni esempi.
}
il video di un fenomeno fisico riprodotto \emph{in reverse} (ovvero ribaltando
l'asse temporale). Il fenomeno fisico rappresentato dovrebbe essere scelto
in modo che sia difficile (o impossibile) distinguere se il video viene
riprodotto in reverse oppure no. Il moto di un pendolo semplice,
di un \emph{pendulum wave} o di altri moti periodici costituiscono buoni
esempi.

Nel contesto di questo progetto si è scelto il video in \emph{slow motion}
di un colibrì in volo, che si nutre da un tubicino (Figura \ref{fig:hummingbird}) 
\footnote{
\`E possibile scaricare il video presso
\href{https://github.com/savaroskij/PED1/blob/master/progetto_finale/media/video/Hummingbird.mp4?raw=true}{questa pagina web}
. L'indirizzo del video originale si trova nella bibliografia\cite{hbird}.
}
. Solamente la componente video, e non quella audio, è stata invertita per
aumentare l'effetto di inganno.

\begin{figure}
\centering
  \includegraphics[width=\textwidth]{Hummingbird}
  \caption{Un frame del video di un colibrì in volo che si nutre da un tubicino.}
  \label{fig:hummingbird}
\end{figure}

Il docente in seguito chiede agli studenti di descrivere quanto viene
visualizzato, e solamente infine svela che il video è riprodotto in
reverse. In questo modo l'insegnante ha l'occasione di far notare
allo studente, e lo esplicita, che per studiare qualsiasi fenomeno fisico
occorrono chiari riferimenti spaziali e temporali.

\section{Extend}
\begin{itemize}
\item \textbf{Tempo richiesto in aula:} 30\textsuperscript{$\prime$}
\item \textbf{Materiale:} Computer, Oggetti domestici
\end{itemize}

In questa fase si proprone un attività mirata ad estendere la consapevolezza
degli studente riguardo ad alcuni concetti presentati precedentemente.
In particolare si desidera estendere l'idea di posizione, che lo studente ha maturato,
allo spazio tridimensiosale.

L'insegnante chiede a ciascuno studente di scegliere un oggetto nella stanza e di
spiegare alla classe, con le proprie parole, dove questo eggetto si trovi.
Lo studente potrebbe utilizzare altri oggetti nella stanza come riferimenti spaziali,
dicendo ad esempio ``L'oggetto si trova accanto alla finestra'', oppure
``L'oggetto si trova sopra il tavolo''; o forse potrebbe utilizzare se stesso come
riferimento affermando ``L'oggetto si trova alla mia destra, in alto''.
A questo punto il docente chiede di scegliere altri due o tre oggetti,
e fa la stessa richiesta di localizzazione. Lo studente si renderà
conto dell'impossibilità di creare nella propria mente un'idea della stanza
degli altri compagni e della posizione di tutti gli oggetti, senza che
costoro esplicitino un riferimento spaziale univoco, e per ogni oggetto indichino tre distanze
misurate o stimate dal riferimento scelto.
L'insegnante dovrebbe guidare gli studenti in questo processo ponendo domande e fornendo suggerimenti
come: ``Dove si trova l'oggetto A rispetto all'oggetto B? E rispetto all'oggetto C?'',
oppure ``Dire che l'oggetto A si trova un metro sopra l'oggetto B è sufficiente
a far comprendere al tuo compagno la posizione dell'oggetto B all'interno della stanza?''
e ancora ``Forse potremmo dire che l'ogetto B si trova sopra l'oggetto A, ma anche alla sua destra
e qualche metro più avanti''.
Al termine di questo processo, il docente ufficializza la conoscenza acquisita disegnando
\footnote{
Il professore può disegnare su una lavagna e filmarsi durante l'operazione, oppure
utilizzare una tavolatta grafica o una lavagna virtuale (presente in software
come Zoom\textsuperscript{\textregistered}, ad esempio)
}          
un diagramma cartesiano tridimensionale, collocando alcuni oggetti al suo interno
e osservando che la posizione dell'oggetto preso come riferimento per tutti gli altri
si chiama \emph{origine} e le tre distanze sono rappresentate da tre numeri lungo
gli assi, chiamati \emph{coordinate cartesiane} tridimensionali.

\chapter{Accelerazione, Moto Rettilineo Uniformemente Accelerato}
\section{Explain}\label{a_explain}

\section{Extend}
\begin{itemize}
\item \textbf{Tempo richiesto in aula:} 60\textsuperscript{$\prime$}
\item \textbf{Materiale:} Smartphone, Computer, Oggetti domestici
\end{itemize}

In questa fase si desidera che lo studente estenda la sua consapevolezza
riguardo il concetto di accelerazione. La letteratura documenta una difficoltà
in una frazione significativa degli studenti nel distinguere tra velocità
e accelerazione. Tale difficoltà si manifesta in particolare nello studio
del fenomeno conosciuto come \emph{Top of the Flight}, o in generale
in tutti i casi nei quali si verifica che la velocità istantanea sia nulla
e contemporaneamente l'accelerazione non lo sia\cite{arons1997teaching}
\cite{trowbridge1981investigation}.

Gli studenti hanno già fatto esperienza di questa situazione nella Sezione
\ref{a_explain}, durante l'analisi dell'esperimento della bicicletta che
scende da una rampa. Quando la bicicletta si trova ferma in cima alla rampa
e inizia la sua discesa, per un solo istante (quello iniziale) la sua velocità
è nulla, mentre la sua accelerazione ha un valore non nullo, che rimmarrà costante
lungo tutta la discesa.

L'insegnante presenta alla classe un surrogato in scala dell'esperimento della
bicicletta, perché possa essere studiato con l'applicazione per smartphone
\emph{Phyphox}
\footnote{
          Phyphox è un applicazione \emph{Free Software} per dispositivi mobili
          (Sistemi supportati Android e IOS) che trasforma il proprio
          smartphone in un piccolo laboratorio di fisica. Maggiori
          informazioni e documentazione sono reperibili presso
          \href{https://phyphox.org/}{questa pagina web}.
         }
. In principio il docente presenta agli studenti l'app, mostrando loro dove
reperirla e le sue funzioni, in particolare lo strumento di misura
\emph{Acceleration (Whitout g)} (Figura \ref{fig:phyphox}) e la funzione
\emph{Save experiment state}.
L'insegnate dovrebbe mostrare un rapido, ma efficave esempio pratico
di utilizzo di questo strumento, per esempio attivando la funzione
di registrazione dati, lanciando in aria il telefono, riprendendolo
al volo e mostrando il risultato agli studenti.

\begin{figure}[H]
\centering
  \includegraphics[width=\textwidth]{phyphox}
  \caption{La schermata dello strumento \emph{Acceleration (Whitout g)}
           nell'applicazione Phyphox.}
  \label{fig:phyphox}
\end{figure}

In seguito il professore assegna agli studenti un compito da eseguire
a casa: indica agli studenti di scegliere una superfice inclinata
e priva di attrito,
\footnote{
          Si ricorda che gli studenti di una classe prima liceo
          non hanno familiarità con il concetto di attrito,
          dunque il professore dovrebbe utilizzare termini differenti
          come ``Superficie inclinata e liscia''.
         }
e di far scivolare il telefono lungo la superficie, attivando
la registrazione dati dello strumento appena mostrato.
Il compito di scegliere l'inclinazione del piano, il modo in cui
collocare il telefono sopra di esso, e altre osservazioni pratiche
sono lasciate allo studente.

Durante la lezione successiva l'insegnate chiede agli studenti
di condividere la propria esperienza e di spiegare l'interpretazione
che questi danno ai grafici prodotti dall'applicazione.
Il docente dovrebbe aver riprodotto l'esperienza, in modo
da poter presentare agli studenti i propri risultati in una
forma più adatta alla comprensione degli stessi.
Si ritiene che grafici prodotti da Phyphox siano utili,
in quanto immediati, ma di difficile interpretazione
per uno studente. Il docente potrà esportare i propri
dati nel formato preferito grazie alla funzione
\emph{Export Data}, e in seguito analizzarli con uno
strumento di propria scelta.
Una possibile analisi è mostrata nella Figura \ref{fig:a_phyphox}.


\begin{figure}[H]
\centering
  \includegraphics[width=\textwidth]{a_phyphox}
  \caption{Il plot, realizzato con il modulo \emph{Matplotlib}
           del linguaggio di programmazione \emph{Python},
           mostra l'accelerazione lungo il piano misurata da
           Phyphox. I dati sono stati mediati nel tempo
           per rimuovere il rumore e un fit lineare è stato
           eseguito sulla coda.
          }
  \label{fig:a_phyphox}
\end{figure}

Il professore aiuta gli studenti ad analizzare il grafico ponendo
alcune domande: ``Riuscite ad individuare nel grafico l'istante
in cui il telefono è stato lasciato libero di scivolare?'',
``Quando vale l'accelerazione prima che il telefono inizi a
scivolare?'', ``Quanto vale l'accelerazione nell'istante
in cui il telefono viene lasciato scivolare, ma è ancora fermo,
ovvero la sua velocità vale zero?'', ``E mentre il telefono sta
scivolando?''. In questo modo egli guida gli studenti verso
la realizzazione che un corpo avente velocità nulla ad un certo
istante, non necessariamente deve avere anche un'accelerazione nulla.

\appendix
\chapter{Analisi Dati}
\section{Moto Rettilineo Uniforme}

\begin{figure}[H]
\centering
  \includegraphics[width=\textwidth]{fit_marble}
  \caption{Fit lineare dei dati per l'esperimento della biglia che cade nel detersivo,
           un esempio di moto rettilineo uniforme.}
  \label{fig:fit_marble}
\end{figure}

\begin{table}[H]
  \renewcommand{\arraystretch}{1.5}
  \centering
  \begin{tabular}{ | c | }
    \hline
    $v$ [$m/s$] \\
    \hline
    $0.01180\pm0.00010$ \\
    \hline
  \end{tabular}
  \caption{Valore atteso ed errore della velocità da fit con il metodo dei minimi
           quadrati.}
  \label{tab:fit_marble}
\end{table}

\section{Moto Rettilineo Uniformemente Accelerato}

\begin{figure}[H]
  \centering
  \includegraphics[width=\textwidth]{fit_bike}
  \caption{Fit lineare dei dati per l'esperimento della bicicletta che scende da una rampa,
           un esempio di moto rettilineo uniformemente accelerato.}
  \label{fig:fit_bike}
\end{figure}

\begin{table}[H]
  \renewcommand{\arraystretch}{1.5}
  \centering
  \begin{tabular}{ | c | c | c | }
    \hline
    $a$ [$m/s^2$] &  $\theta$ [deg] & pendendenza [\%] \\
    \hline
    $0.706\pm0.015$ & $4.13\pm0.09$ & $7.21\pm0.15$ \\
    \hline
  \end{tabular}
  \caption{Valore atteso ed errore dell'accelerazione, dell'angolo di inclinazione della rampa
           e della pendenza percentuale da fit con il metodo dei minimi quadrati.}
  \label{tab:fit_bike}
\end{table}

\bibliography{bibliografia}{}
\bibliographystyle{plain}
\addcontentsline{toc}{chapter}{Riferimenti}

\end{document}
